% template.tex
\documentclass[12pt, a4paper]{article}
\usepackage{lab-report}

% Define the info of the lab:
% \def\RPTTITLE{Template Report}
\reporttitle{Lab}{1}{Introductory Report}
\courseinfo{ECE 385}{Digital Systems Laboratory}{Spring 2023}
\classinfo{LE1}{Thursday 18:00 -- 20:50}{D231 Teaching Lab}
\logo{logo.png}

\members{
    \centering

    \Large \textbf{San Zhang (张\hspace{1em}三)}\\
    \Large 8888888888

    \Large \textbf{Haoran Meng (孟浩然)}\\
    \Large 9999999999
}

\begin{document}
    % Insert the titlepage
    \maketitlepage

    {\bfseries The following contents are copied from a blank layout sheet provided on iZJU Blackboard, which can be accessed through \url{https://learn.intl.zju.edu.cn/bbcswebdav/pid-51159-dt-content-rid-503119_1/courses/UIUC-ECE385-2110-M/385LabSample%281%29.pdf} }

    \section{Purpose of Circuit}
    This is where you should briefly (a few sentences) describe the purpose of this lab, i.e. is 
    this lab about flip-flops? Memory? Computation unit? How is this lab useful in the 
    context of a processor architecture?

    \section{Description of Circuit}
    This is where you should explain the operation of your circuit in detail. Include a 
    description of any logic that you have designed, and explain \textbf{how you came up with 
    your design}. This means things like K-Maps, state diagrams, truth tables, etc... A rule 
    of thumb is that for anything you have done towards the designing your circuit, you 
    should include in your lab report in an \underline{organized} fashion that \underline{makes sense} (we've seen 
    many lab reports with scribbled pages as attachment. This is not a good presentation of 
    your work).

    Do not just list the components in your circuit (e.g. ``this design uses 4 NAND gates and 
    an inverter to perform the function; we drive input B with this signal, and this is what we 
    got for the output''). This is the job of the logic diagrams and the layout sheet. Rather, a 
    well-informed circuit description should be a more detailed explanation of the `Purpose 
    of Circuit', i.e., if you are working on a memory system, what components does the 
    system need (e.g., registers for the actual storage; MUXs for data selection; control unit 
    for read/write request)? How are the registers arranged (i.e., in series/parallel? How 
    many of them? How large is each storage space)? How do you operate the system (e.g., 
    the circuit is manually resetted. If we wish to store data X into address Y, we need to 
    operate control Z, then the circuit goes through processes A→D to complete the storing 
    action...)

    \section{High Level Block Diagram}
    For more complex designs, divide your design into blocks and include a top-level block 
    diagram showing only the major components of your design and their interconnections. 
    Block diagrams are higher level representation of your logic diagram. It divides your 
    logic diagram into large 'chunks' into black boxes (e.g. Fig. 1 on pp. 3.2), where each box 
    serves a meaningful purpose in the large picture of your design. The block diagram does 
    not give you the details of the implementation at all, but its purpose is to introduce your 
    design and your circuit in an abstract way, so people will be able to fully grasp the big 
    idea in a short amount of time. Please note that block diagrams are needed in your lab 
    reports. The more logical you split up your design modules, the more likely you can 
    clear-cut them apart for wiring and debugging purposes (it also enhances the organization 
    and hence the readability of your design).

    \section{Logic Diagrams}
    Include a logic diagram showing the gate-level layout of your circuit. Be sure to label all 
    inputs, outputs, pin numbers, as well as the chip locations on the protoboard (B1, C3, etc. 
    in coordination with the component layout sheet). For an example see Figure 9 on page 
    GG.21 of the ECE 385 lab manual.

    \section{State Diagrams and Tables}
    For any design that uses a state machine, you should include a state diagram and, 
    optionally, transition tables to show how your state machine works.

    \section{Component Layout}
    Include a component layout of your circuit using the sheets provided in your lab manual. 
    You should include the layout of each circuit (for two or more circuits for a single lab, 
    please draw clear lines to separate them). For an example, see Figure 8 on page GG.20 of 
    the ECE 385 lab manual.

    For each chip of each circuit, you should draw out the pins (different chips will have 
    different numbers of pins on each side). \underline{DO NOT} label every single pin in your design, 
    as it is impossible for anyone to read. \underline{DO} label fundamental pins such as external IOs to 
    the IO box, VDD, and GND. Remember to write the chip number in each component.

    \section{Answers to Pre-Lab Questions}
    You should include the answers to all of the pre-lab questions in your lab report. Answer 
    these questions before your come into the lab to do the experiment, but correct the 
    answers if you discover during or after the lab that you made a mistake.

    ... 


\end{document}